\section{Описание решения}
В этой главе подробно рассмотрен весь алгоритм распознавания речевых команд. Для удобства понимая названия параграфов расположены в том же порядке что и стадии работы в самом алгоритме.

\subsection{Предобработка}
\subsubsection{Удаление постоянной составляющей}
Постоянная составляющая (DC-offset) - это смещение амплитуды сигнала на некоторую постоянную величину. Возникает это в аналого-цифровом сигнале из-за разницы напряжения между звуковой картой и устройством ввода. Для того, чтобы избавиться от неё, необходимо вычесть из каждого значения амплитуды среднее арифметическое всех значений амплитуд по формуле:

\begin{equation}
\overline{x}_i=x_i - \sum_{j=0}^{p} x_j,
\end{equation}
где $x_i$ - новое значение амплитуды с индексом $i$, $\overline{x}_i$ - новое значение амплитуды с индексом $i$, $p$ - количество амплитудных значений в звуковой дорожке. 


\subsubsection{Выделение начальной и конечной точек слова}

\subsection{Выделение речевых признаков}



\subsection{Распознавание речевых команд}
\subsubsection{Описание входных и выходных данных модели}
\subsubsection{Архитектура нейронной сети}
\subsubsection{}


\subsection{Распознавание речевых команд}