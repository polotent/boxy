\section{Описание решения}
В этой главе подробно рассмотрен весь алгоритм распознавания речевых команд. Для удобства понимая названия параграфов расположены в том же порядке, что и этапы в самом алгоритме.

\subsection{Предобработка}
Каждая команда - это звуковой wav файл. В каждом файле - набор амплитудных значений, которые были получены в результате записи команд дикторами. 

\subsubsection{Нормализация сигнала}
В начале проводится нормализация амплитуд. Каждое значение амплитуды приводится к такому значению, чтобы минимум среди всех амплитуд звуковой дорожки был в 0, а максимум среди всех - в 1 по формуле:
\begin{equation}
	\overline{x}_i=\dfrac{x_i}{\max_{j} |x_j|},~i=\overline{1, p},~j \in [1, p]
\end{equation}
где $x$ - значение амплитуды, $\overline{x}$ - новое значение амплитуды, $p$ - количество амплитудных значений в звуковой дорожке.

Таким образом все значения амплитуд принимают значения в диапазоне $[0,1]$.

\subsubsection{Удаление постоянной составляющей}
Постоянная составляющая (DC-offset) - это смещение амплитуды сигнала на некоторую постоянную величину. Возникает это в аналого-цифровом сигнале из-за разницы напряжения между звуковой картой и устройством ввода. Данный эффект является помехой, от которой нужно избавиться. Для этого необходимо вычесть из каждого значения амплитуды среднее арифметическое всех значений амплитуд по формуле:
\begin{equation}
\overline{x}_i=x_i - \sum_{j=1}^{p} x_j,~i=\overline{1, p}
\end{equation}
где $x$ - значение амплитуды полученное на этапе нормализации, $\overline{x}$ - новое значение амплитуды, $p$ - количество амплитудных значений в звуковой дорожке.

\subsubsection{Выделение начальной и конечной точек слова}
Каждая звуковая дорожка содержит в себе помимо фрагментов звукового сигнала - команды ещё и фрагменты тишины. Очень важно отделить звуковой сигнал от фрагментов тишины, т. к. именно он несёт в себе всю информацию о команде. 

Для того, чтобы выделить звуковой сигнал и <<обрезать>> тишину в начале и в конце записи, используется алгоритм, описанный статье \cite{SignalPreprocessing}. 
Каждая звуковая дорожка разбивается на фреймы - наборы амплитуд, каждый длительностью 20 мс. Начала фреймов расположены с периодичностью 10 мс. Таким образом, фреймы пересекаются между собой. Это обеспечивает целостность обработки звукового сигнала, т.е. позволяет не упустить важные фонемообразующие особенности.

Затем для каждого фрейма вычисляется мгновенная энергия:
\begin{equation}
E_k = \sum_{m=1}^{N} x_{k_m}^2,~k=\overline{1,z}
\end{equation}
где $z$ - количество фреймов для конкретной звуковой записи, N - длина одного фрейма (количество амплитуд в одном фрейме).

Мгновенная энергия имеет один значительный недостаток. У неё очень большая чувствительность к относительно большим значениям амплитуды из-за возведения во вторую степень. Это ведёт к искажению соотношений отсчётов звукового сигнала между друг другом. Поэтому функция мгновенной энергии переопределяется как:
\begin{equation}
\label{eq:instant_energy}
E_k = \sum_{m=1}^{N} |x_{k_m}|,~k=\overline{1,z}
\end{equation}

После того, как посчитаны мгновенные энергии для каждого фрейма, вычисляется нижнее и верхнее пороговые значения:
\begin{equation}
\begin{aligned}
& I_1 = 0.03 \cdot (MX - MN) + MN \\
& I_2 = 4 \cdot MN \\
& ITL = min(I_1,~I_2)\\
& ITU = 10 \cdot ITL
\end{aligned}
\end{equation}
где $MN$, $MX$ - минимум и максимум мгновенной энергии среди всех фреймов соответственно, $ITL$, $ITU$ - нижнее и верхнее пороговое значение.

Происходит поиск фрейма, с которого начинается слово с самого первого фрейма. Фрейм, в котором значение мгновенной энергии превышает $ITL$, предварительно помечается как начало слова. Затем начиная с этого помеченного фрейма происходит поиск фрейма, в котором значение мгновенной энергии превышает $ITU$. Если значение мгновенной энергии для какого-то фрейма во время последнего поиска меньше $ITL$, то этот фрейм становится предварительным началом слова. 

Аналогично происходит поиск конца слова в звуковой дорожке. Только поиск по фреймам происходит не с начала сигнала, а с конца.

После этого этапа имеются два предварительно помеченных фрейма $m_1, m_2$  - начало и конец слова в звуковом файле.

Функция мгновенной энергии, определённая формулой \eqref{eq:instant_energy} хорошо справляется с отделением звонких звуков от тишины. Но вот глухие она отделяет плохо. Поэтому используется вторая характеристика для доопределения начала и конца слова - число переходов через ноль. Это количество таких случаев, когда соседние отсчёты (значения амплитуд) имеют противоположные знаки. Определяется формулой:
\begin{equation}
	Z_k = \dfrac{1}{2} \sum_{m=2}^{N} |sgn(x_{k_{m-1}}) - sgn(x_{k_m})|,~k=\overline{1,z}
\end{equation}
Подразумевается, что первые 100 мс звуковой записи - это тишина, и речь начинается позднее.

Вычисляется среднее значение переходов через ноль в течение первых 100 мс \eqref{eq:izc},  среднее квадратическое отклонение количества переходов через ноль в течение первых 100 мс \eqref{eq:deviation}:
\begin{align}
	\label{eq:izc}
	&IZC = \dfrac{1}{z} \sum_{k=1}^{z} Z_k \\
	\label{eq:deviation}
	&\sigma_{IZC} = \sqrt{\dfrac{1}{z} \sum_{k=1}^{z} (Z_k - IZC)^2}
\end{align}
а затем пороговую функцию числа переходов через ноль по формуле:
\begin{equation}
	IZCT = min(IF,~IZC + 2 \sigma_{IZC}),
\end{equation}
где $IF$ - фиксированное количество переходов через ноль (25 пересечений за 10 мс).

Происходит уточнение точек начала и конца слова в звуковой дорожке. Начиная от фрейма $m_1$ влево происходит поиск фреймов, у которых число переходов через ноль выше порогового значения. Поиск происходит всего на расстоянии 25 фреймов, так как производится уточнение границ слова. Если пороговое значение было превышено 3 или более раз, то фрейм $r_1$, где это произошло впервые, помечается как начало слова.

Аналогично от фрейма $m_2$ происходит поиск вправо для уточнения точки конца слова.

На выходе этого алгоритма - 2 помеченных фрейма $r_2$. В итоге, сигнал обрезается, и остаётся только речевая команда в виде набора фреймов $[r_1, ... , r_2]$.  

\subsection{Выделение речевых признаков}
Для того, чтобы выделить речевые признаки, используется алгоритм MFCC \cite{MFCC}. Он является одним из стандартных подходов к решению поставленной задачи. Состоит MFCC из нескольких шагов:
\begin{enumerate}
	\item Для каждой звуковой дорожки проделать шаги:
	\begin{enumerate}
		\item Разбить сигнал на фреймы и для каждого фрейма проделать шаги:
		\begin{enumerate}
			\item Получить спектр сигнала
			\item  
		\end{enumerate}
		\item Объединить MFCC векторы коэффициентов в матрицу.
	\end{enumerate}

	\item 
\end{enumerate}
  
\subsubsection{}

\subsection{Распознавание речевых команд}
\subsubsection{Описание входных и выходных данных модели}
\subsubsection{Архитектура нейронной сети}
\subsubsection{}


\subsection{Распознавание речевых команд}