\section{Описание решения}
В этой главе подробно рассмотрен весь алгоритм распознавания речевых команд. Для удобства понимая названия параграфов расположены в том же порядке, что и этапы в самом алгоритме.

\subsection{Предобработка}
Каждая команда - это звуковой wav файл. В каждом файле - набор амплитудных значений, которые были получены в результате записи команд дикторами. 

\subsubsection{Нормализация сигнала}
В начале проводится нормализация амплитуд. Каждое значение амплитуды приводится к такому значению, чтобы минимум среди всех амплитуд звуковой дорожки был в 0, а максимум среди всех - в 1 по формуле:
\begin{equation}
	\overline{x}_i=\dfrac{x_i}{\max_{j} |x_j|},~i=\overline{0, p},~j \in [0, p]
\end{equation}
где $x$ - значение амплитуды, $\overline{x}$ - новое значение амплитуды, $p$ - количество амплитудных значений в звуковой дорожке.

Таким образом все значения амплитуд принимают значения в диапазоне $[0,1]$.

\subsubsection{Удаление постоянной составляющей}
Постоянная составляющая (DC-offset) - это смещение амплитуды сигнала на некоторую постоянную величину. Возникает это в аналого-цифровом сигнале из-за разницы напряжения между звуковой картой и устройством ввода. Данный эффект является помехой, от которой нужно избавиться. Для этого необходимо вычесть из каждого значения амплитуды среднее арифметическое всех значений амплитуд по формуле:
\begin{equation}
\overline{x}_i=x_i - \sum_{j=0}^{p} x_j,~i=\overline{0, p}
\end{equation}
где $x$ - значение амплитуды полученное на этапе нормализации, $\overline{x}$ - новое значение амплитуды, $p$ - количество амплитудных значений в звуковой дорожке.

\subsubsection{Выделение начальной и конечной точек слова}
Каждая звуковая дорожка содержит в себе помимо фрагментов звукового сигнала - команды ещё и фрагменты тишины. Очень важно отделить звуковой сигнал от фрагментов тишины, т. к. именно он несёт в себе всю информацию о команде. 

Для того, чтобы выделить звуковой сигнал и <<обрезать>> тишину в начале и в конце записи, используется алгоритм, описанный статье \cite{SignalPreprocessing}. 
Каждая звуковая дорожка разбивается на фреймы - наборы амплитуд, каждый длительностью 20 мс. Начала фреймов расположены с периодичностью 10 мс. Таким образом, фреймы пересекаются между собой. Это обеспечивает целостность обработки звукового сигнала, т.е. позволяет не упустить важные фонемообразующие особенности.

Затем для каждого фрейма вычисляется мгновенная энергия:
\begin{equation}
E_k = \sum_{m=0}^{N-1} x_{k_m}^2,~k=\overline{0,z}
\end{equation}
где $z+1$ - количество фреймов для конкретной звуковой записи, N - длина одного фрейма (количество амплитуд в одном фрейме).

Мгновенная энергия имеет один значительный недостаток. У неё очень большая чувствительность к относительно большим значениям амплитуды из-за возведения во вторую степень. Это ведёт к искажению соотношений отсчетов звукового сигнала между друг другом. Поэтому функция мгновенной энергии переопределяется как:
\begin{equation}
\label{eq:instant_energy}
E_k = \sum_{m=0}^{N-1} |x_{k_m}|,~k=\overline{0,z}
\end{equation}

После того, как посчитаны мгновенные энергии для каждого фрейма, вычисляется нижнее и верхнее пороговые значения:
\begin{equation}
\begin{aligned}
& I_1 = 0.03 \cdot (MX - MN) + MN \\
& I_2 = 4 \cdot MN \\
& ITL = min_{I_1, I_2} \\
& ITU = 10 \cdot ITL
\end{aligned}
\end{equation}
где $MN$, $MX$ - минимум и максимум мгновенной энергии среди всех фреймов соответственно, $ITL$, $ITU$ - нижнее и верхнее пороговое значение.

Происходит поиск фрейма, с которого начинается слово с самого первого фрейма. Фрейм, в котором значение мгновенной энергии превышает $ITL$, предварительно помечается как начало слова. Затем начиная с этого помеченного фрейма происходит поиск фрейма, в котором значение мгновенной энергии превышает $ITU$. Если значение мгновенной энергии для какого-то фрейма во время последнего поиска меньше $ITL$, то этот фрейм становится предварительным началом слова. 

Аналогично происходит поиск конца слова в звуковой дорожке. Только поиск по фреймам происходит не с начала сигнала, а с конца.

После этого этапа имеются две предварительные метки - начало и конец слова в звуковом файле.

Функция мгновенной энергии, определённая формулой \eqref{eq:instant_energy} хорошо справляется с отделением звонких звуков от тишины. Но вот глухие она отделяет плохо. Поэтому используется вторая характеристика для доопределения начала и конца слова - число переходов через ноль. Это количество таких случаев, когда соседние отсчеты (значения амплитуд) имеют противоположные знаки. Определяется формулой:
\begin{equation}
	Z_k = \dfrac{1}{2} \sum_{m=1}^{N-1} |sgn(x_{m-1}) - sgn(x_m)|,~k=\overline{0,z}
\end{equation}


\subsection{Выделение речевых признаков}



\subsection{Распознавание речевых команд}
\subsubsection{Описание входных и выходных данных модели}
\subsubsection{Архитектура нейронной сети}
\subsubsection{}


\subsection{Распознавание речевых команд}