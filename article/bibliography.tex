\begin{thebibliography}{1}
\bibitem{Audrey}
Davis K. N., Biddulph R., Balashek S. Automatic recognition of spoken digits // The Journal of the Acoustical Society of America, 1952. Vol. 24, No 6. P. 637-642 

\bibitem{Harpy}
Newell A. Harpy, production systems and human cognition // Research Showcase @ Carnegie Mellon University, 1978

\bibitem{spectra}
Plomp R., Pols L. C. W., van der Geer J.P. Dimensional analysis of vowel spectra // The Journal of the Acoustical Society of America, 1967, Vol. 41, P. 707-712 

\bibitem{cepstra}
Bogert B. P., Healy M. J. R., Tukey J. W. The Quefrency Alanysis of Time Series for Echoes: Cepstrum, Pseudo Autocovariance, Cross-Cepstrum and Saphe Cracking // Proceedings of the Symposium on Time Series Analysis, 1963, Ch. 15, P. 209-243

\bibitem{MFCC}
Davis S., Mermelstein P. Comparison of parametric representations for monosyllabic word recognition in continuously spoken sentences // IEEE Transactions on Acoustics, Speech, and Signal Processing, 1980, Vol. ASSP-28, No. 4, P. 357-366

\bibitem{SignalPreprocessing}
Rabiner L.R., Sambur, M.R. An Algorithm for Determining the Endpoints of Isolated Utterances // The Bell System Technical Journal, 1975, Vol. 54, No. 2, P. 297-315

\bibitem{explainingLog}
Аксёнов О.Д. Метод мел-частотных кепстральных коэффициентов в задаче распознавания речи // 55-я юбилейная научная конференция аспирантов, магистрантов и студентов БГУИР, 2019, С. 45-46

\bibitem{Perceptron}
Rosenblatt, F. Principles of Neurodynamics: Perceptrons and the Theory of Brain Mechanisms. Spartan Books, Washington DC, 1961

\bibitem{Mel}
O'Shaughnessy D. Speech communication: human and machine, Addison-Wesley, 1987, P. 150

\bibitem{CNN}
LeCun Y., Bengio Y.. Convolutional networks for images, speech, and time-series // The Handbook of Brain Theory and Neural Networks, 1995, MIT 
\end{thebibliography}