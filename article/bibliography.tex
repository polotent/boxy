\begin{thebibliography}{1}
% \addcontentsline{toc}{section}{Список литературы}
\bibitem{Audrey}
Davis K. N., Biddulph R., Balashek S. Automatic recognition of spoken digits // The Journal of the Acoustical Society of America, 1952. Vol. 24, No 6. P. 637-642 

\bibitem{PerceptronArticle}
Rosenblatt, F. The perceptron: A probabilistic model for information storage and organization in the brain // Psychological Review, 1958, Vol. 65(6), P. 386–408

\bibitem{PerceptronBook}
Rosenblatt, F. Principles of Neurodynamics: Perceptrons and the Theory of Brain Mechanisms. Spartan Books, Washington DC, 1961

\bibitem{Cepstrum}
Bogert B. P., Healy M. J. R., Tukey J. W. The Quefrency Alanysis of Time Series for Echoes: Cepstrum, Pseudo Autocovariance, Cross-Cepstrum and Saphe Cracking // Proceedings of the Symposium on Time Series Analysis, 1963, Ch. 15, P. 209-243

\bibitem{Spectrum}
Plomp R., Pols L. C. W., van der Geer J.P. Dimensional analysis of vowel spectra // The Journal of the Acoustical Society of America, 1967, Vol. 41, P. 707-712 

\bibitem{SignalPreprocessing}
Rabiner L.R., Sambur, M.R. An Algorithm for Determining the Endpoints of Isolated Utterances // The Bell System Technical Journal, 1975, Vol. 54, No. 2, P. 297-315

\bibitem{Harpy}
Newell A. Harpy, production systems and human cognition // Research Showcase @ Carnegie Mellon University, 1978

\bibitem{CeptrumExplanation}
Оппенгейм А. В., Шафер Р. В. Цифровая обработка сигналов / Под ред. С. Я. Шаца, М.: Связь, 1979. С. 416

\bibitem{MFCC}
Davis S., Mermelstein P. Comparison of parametric representations for monosyllabic word recognition in continuously spoken sentences // IEEE Transactions on Acoustics, Speech, and Signal Processing, 1980, Vol. ASSP-28, No. 4, P. 357-366

\bibitem{Mel}
O'Shaughnessy D. Speech communication: human and machine, Addison-Wesley, 1987, P. 568

\bibitem{CNN}
LeCun Y., Bengio Y.. Convolutional networks for images, speech, and time-series // The Handbook of Brain Theory and Neural Networks, 1995, MIT 
\end{thebibliography}