\specialsection{Постановка задачи}
В данной работе ставится задача распознавания речи. Под распознаванием речи понимается нахождение отображения множества речевых команд во множество соответствующих им скалярных меток. 

Пусть $X=\{x_0, ..., x_{p-1}, ...\}$ - бесконечное множество речевых команд. Оно состоит из векторов амплитуд $x_i=\{x_i^0, ..., x_i^{k_i}\},~i=\overline{0,\inf}$. Здесь $k_i$ - количество записанных амплитуд в $i$-й речевой команде. Известны значения первых $p$ элементов, обозначаемых через $\widehat{X}=\{x_0, ..., x_{p-1}\}$.  

Пусть $Y=\{y_0, ..., y_{g-1}\}$ - известное конечное множество уникальных скалярных меток. Известно отображение $\widehat{Z}=\widehat{X} \rightarrow Y$ для первых $p$ речевых команд.

Необходимо разработать алгоритм, который бы строил отображение $Z = X \rightarrow Y$.

Поставленная задача разбивается на следующие подзадачи:
\begin{itemize}[leftmargin=2cm]
	\item разработка и реализация алгоритма предобработки исходных данных;
	\item разработка и реализация алгоритма распознавания речевых команд;
	\item проведение вычислительных экспериментов и исследование работоспособности и эффективности алгоритма распознавания речевых команд;
	\item выводы об эффективности исследуемого подхода к решению задачи распознавания речи для прикладного применения.
\end{itemize}

Под множеством речевых команд понимается набор слов-команд для управления медиаплеером. Базовый набор команд составляет необходимый минимум для управления программным интерфейсом. В рамках данной работы сам интерфейс управления медиаплеером не рассматривается.