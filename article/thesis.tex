\specialsection{Постановка задачи}
Пусть $X=\{x_0, ..., x_{p-1}, ...\}$ - множество объектов речи. Оно состоит из векторов амплитуд $x_i=\{x_i^0, ..., x_i^{k_i}\},~i=\overline{0,\inf}$. Здесь $k_i$ - количество записанных амплитуд в $i$-м объекте. Само по себе множество объектов бесконечно, однако известны значения первых $p$ элементов. Обозначим их через $\widehat{X}=\{x_0, ..., x_{p-1}\}$.  

Пусть $Y=\{y_0, ..., y_{p-1}, ...\}$ - множество скалярных меток, а $\widehat{Y}=\{y_0, ..., y_{p-1}\}$ - множество известных скалярных меток для первых $p$ объектов речи. Таким образом известно отображение $\widehat{Z}=\widehat{X} \rightarrow \widehat{Y}$, описываемое парами значений $\widehat{Z}=\{(x_0, y_0), ..., (x_{p-1}, y_{p-1})\}$.

Необходимо разработать алгоритм, который бы строил отображение $Z = X \rightarrow Y$.


Для того, чтобы решить поставленную задачу, необходимо разбить ее на следующие подзадачи и последовательно решить их:
\begin{itemize}[leftmargin=2cm]
\item провести предобработку первоначальных данных, содержащих звуковой сигнал в виде наборов амплитуд;
\item разработать алгоритм распознавания объектов речи;
\item реализовать алгоритм распознавания объектов речи;
\item провести вычислительные эксперименты и выяснить, какой метод решения задачи является наиболее эффективным.
\end{itemize}


В дальнейшем под объектом речи понимается отдельно взятое слово. В контексте этой работы словом является речевая команда, которая произносится на английском языке в микрофон. Базовый набор команд составляет необходимый минимум для управления программным интерфейсом медиаплеера. В рамках данной работы сам интерфейс не рассматривается.