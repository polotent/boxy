\documentclass[14pt]{article}

\usepackage[utf8]{inputenc}
\usepackage[russian]{babel}

\usepackage{setspace}
\setstretch{1.5}

\usepackage{indentfirst}
\setlength{\parindent}{1.25cm}

\usepackage{geometry}
\usepackage{graphicx}
\usepackage{tocloft}
\usepackage{hyperref}

\renewcommand{\cftsecleader}{\cftdotfill{\cftdotsep}}
\renewcommand{\contentsname}{\centering Содержание}

\begin{document}
\newgeometry{left=30mm, top=20mm, right=15mm, bottom=20mm, nohead}
\begin{titlepage}
\begin{center}
Санкт--Петербургский государственный университет \\
\textbf{Кафедра компьютерного моделирования и многопроцессорных систем}
\vspace{55mm} \\
\textbf{\large Мирошниченко Александр Сергеевич} \\[10mm]
\textbf{\large Выпускная квалификационная работа бакалавра} \\[10mm]
\textbf{\large Разработка системы распознавания речевых команд при помощи методов машинного обучения} \\
Направление 01.03.02 \\
«Прикладная математика и информатика»\\[30mm]
\begin{flushright}
{Научный руководитель,} \\
кандидат физ.-мат. наук, \\доцент \\Козынченко~В. А. 
\end{flushright}
\vfill 
{Санкт-Петербург}
\par{2021 г.}
\end{center}
\end{titlepage}
\addtocounter{page}{1}

\tableofcontents
\newpage

\addcontentsline{toc}{section}{Введение}
\section*{Введение}
Какое-то введение \\
И вот новая строка

\addcontentsline{toc}{section}{Постановка задачи}
\section*{Постановка задачи}

\addcontentsline{toc}{section}{Обзор литературы}
\section*{Обзор литературы}

\addcontentsline{toc}{section}{Глава 1. Теоретические сведения}
\section*{Глава 1. Теоретические сведения}

\addcontentsline{toc}{section}{Глава 2. Описание решения}
\section*{Глава 2. Описание решения}

\addcontentsline{toc}{section}{Глава 3. Результаты вычислений}
\section*{Глава 3. Результаты вычислений}

\addcontentsline{toc}{section}{Выводы}
\section*{Выводы}

\addcontentsline{toc}{section}{Заключение}
\section*{Заключение}

\addcontentsline{toc}{section}{Список литературы}
\begin{thebibliography}{3}
\bibitem{Aurélien2019}
Aurélien G. Hands-On Machine Learning with Scikit-Learn, Keras, and TensorFlow: Concepts, Tools, and Techniques to Build Intelligent Systems / Aurélien G. --- 2nd Edition --- O'Reilly Media, 2019.
\bibitem{Kailash2019}
Kailash A. Generative Adversarial Networks Projects: Build next-generation generative models using TensorFlow and Keras / Kailash A. ---  Packt Publishing, 2019.
\bibitem{TensorFlowDocs}
Документация TensorFlow [Электронный ресурс]. --- Режим доступа: https://www.tensorflow.org/api\_docs/python/tf
\bibitem{MLportal}
Портал ML Glossary [Электронный ресурс]. --- Режим доступа: https://ml-cheatsheet.readthedocs.io
\bibitem{CourseraCourse}
Курс на платформе Coursera [Электронный ресурс]. --- Режим доступа: https://www.coursera.org/learn/getting-started-with-tensor-flow2
\end{thebibliography}

\addcontentsline{toc}{section}{Приложение}
\section*{Приложение}
Ссылка на репозиторий c кодом: \url{https://gitlab.com/polotent/boxy}
\restoregeometry

\end{document}