\documentclass[14pt]{article}

\usepackage[utf8]{inputenc}
\usepackage[russian]{babel}

\usepackage{setspace}
\setstretch{1.5}

\usepackage{indentfirst}
\setlength{\parindent}{1.25cm}

\usepackage{geometry}
\usepackage{graphicx}
\usepackage{tocloft}
\usepackage{hyperref}
\usepackage{enumitem}
\usepackage{csvsimple}

\begin{document}
\newgeometry{left=30mm, top=20mm, right=15mm, bottom=20mm, nohead}
\begin{titlepage}
\begin{center}
Санкт--Петербургский государственный университет \\
\textbf{Кафедра компьютерного моделирования и многопроцессорных систем}
\vspace{55mm} \\
\textbf{\large Мирошниченко Александр Сергеевич} \\[10mm]
\textbf{\large Выпускная квалификационная работа бакалавра} \\[10mm]
\textbf{\large Разработка системы распознавания речевых команд при помощи методов машинного обучения} \\
Направление 01.03.02 \\
«Прикладная математика и информатика»\\[30mm]
\begin{flushright}
{Научный руководитель,} \\
кандидат физ.-мат. наук, \\доцент \\Козынченко~В. А. 
\end{flushright}
\vfill 
{Санкт-Петербург}
\par{2021 г.}
\end{center}
\end{titlepage}
\addtocounter{page}{1}

\tableofcontents
\newpage

\addcontentsline{toc}{section}{Введение}
\section*{Введение}
\newpage


\addcontentsline{toc}{section}{Постановка задачи}
\section*{Постановка задачи}
\newpage

\addcontentsline{toc}{section}{Обзор литературы}
\section*{Обзор литературы}
\newpage

\addcontentsline{toc}{section}{Глава 1. Теоретические сведения}
\section*{Глава 1. Теоретические сведения}
\newpage

\addcontentsline{toc}{section}{Глава 2. Описание решения}
\section*{Глава 2. Описание решения}
\newpage

\addcontentsline{toc}{section}{Глава 3. Результаты вычислений}
\section*{Глава 3. Результаты вычислений}
Было проведено 3 вычислительных эксперемента для каждого из 2-х типов нейронной сети: MLP и CNN. Структуры приведены на рис 1,2

Датасет состоит из 6 дикторов. Каждый диктор работал с 11 командами : 'back', 'down', 'menu', 'off', 'on', 'open', 'play', 'power', 'stop', 'up', 'volume'.
\\

\begin{tabular}[c]{ | p{2cm} | p{2cm} | p{6cm} | p{4cm} | }
\hline
Диктор & Тип голоса & Кол-во звук. дорожек на каждую команду & Сумм. кол-во звук. дорожек  \\ \hline
speaker1 & Мужской & 50 & 550 \\
speaker2 & Мужской & 40 & 440 \\
speaker3 & Мужской & 40 & 440 \\
speaker4 & Мужской & 40 & 440 \\
speaker5 & Мужской & 50 & 550 \\
speaker6 & Женский & 50 & 550 \\ \hline
\end{tabular}
\\

Первый эксперимент: нейронная сеть обучается на первом дикторе с мужским голосом, тестирование производится на каждом дикторе. 

Второй эксперимент: нейронная сеть обучается на всех дикторах с мужским голосом, тестирование производится на каждом дикторе. 

Третий эксперимент: нейронная сеть обучается на всех дикторах, тестирование производится на каждом дикторе.

Датасет предварительно разделяется на тренировочную и тестовую части. На тренировочную часть отводится 70\% данных диктора, на тестовую часть - 30\%. В процессе тренировки после каждой эпохи тренировочные данные перемешиваются. 15\% тренировочных данных в каждой эпохе - валидационные. В качестве метрики для оценки эффективности была выбрана метрика точности (accuracy), а для валидации - функция потерь категориальной кросс-энтропии (val\_loss). Алгоритм оптимизации - Adam. Максимальное количество эпох - 50. Если значение метрики val\_loss не уменьшается в течение 20 эпох, то обучение останавливается.

Графики обучения для каждого из экспериментов приведены на рисунках.

В конце каждого эксперимента помимо тестирования производится построение матрицы ошибок (confusion matrix) для каждого диктора и для каждого из четырех пороговых значений: 0.5, 0.6, 0.7, 0.8.
\\

\begin{table}
\centering
\csvautotabular[respect underscore=true]{test_summary.csv}
\caption{Результаты вычислений}
\end{table}


all\_speakers = [speaker1, speaker2, speaker3, speaker4, speaker5, speaker6]

all\_male\_speakers = [speaker1, speaker2, speaker3, speaker4, speaker5]

\newpage


\addcontentsline{toc}{section}{Выводы}
\section*{Выводы}
\newpage

\addcontentsline{toc}{section}{Заключение}
\section*{Заключение}
В данной работе:
\begin{itemize}[leftmargin=2cm]
\item Проведена предобработка звуковых дорожек, содержащих команды
\item Разработан алгоритм распознавания речевых команд
\item Реализован алгоритм распознавания речевых команд
\item Проведены вычислительные эксперименты, в результате которых	 показана работоспособность и эффективность работы алгоритма распознавания речевых команд.
\end{itemize}
\newpage

\addcontentsline{toc}{section}{Список литературы}
\begin{thebibliography}{3}
\bibitem{Aurélien2019}
Aurélien G. Hands-On Machine Learning with Scikit-Learn, Keras, and TensorFlow: Concepts, Tools, and Techniques to Build Intelligent Systems / Aurélien G. --- 2nd Edition --- O'Reilly Media, 2019.
\bibitem{Kailash2019}
Kailash A. Generative Adversarial Networks Projects: Build next-generation generative models using TensorFlow and Keras / Kailash A. ---  Packt Publishing, 2019.
\bibitem{TensorFlowDocs}
Документация TensorFlow [Электронный ресурс]. --- Режим доступа: https://www.tensorflow.org/api\_docs/python/tf
\bibitem{MLportal}
Портал ML Glossary [Электронный ресурс]. --- Режим доступа: https://ml-cheatsheet.readthedocs.io
\bibitem{CourseraCourse}
Курс на платформе Coursera [Электронный ресурс]. --- Режим доступа: https://www.coursera.org/learn/getting-started-with-tensor-flow2
\end{thebibliography}
\newpage

\addcontentsline{toc}{section}{Приложение}
\section*{Приложение}
Ссылка на репозиторий c кодом: \url{https://gitlab.com/polotent/boxy}
\restoregeometry

\end{document}