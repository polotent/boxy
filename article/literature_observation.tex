\specialsection{Обзор литературы}
Далее в хронологическом порядке описаны наиболее важные моменты и работы, связанные с поставленной задачей.

Первые шаги в распознавании речи были сделаны в 1952 году. Тогда трое исследователей Bell Labs - Стивен Балашек, Рулон Биддалф и Кей Дэвис - представили публике первый в истории аппарат Audrey, способный распознавать человеческую речь \cite{Audrey}. Это была система, позволявшая распознавать только цифры. 

В 1957 году Фрэнк Розенблатт предложил математическую модель восприятия информации мозгом - персептрон. В своей статье \cite{PerceptronArticle}, а позднее и в своей книге \cite{PerceptronBook}, он описал принципы работы модели. Этот момент можно считать появлением нейронных сетей как таковых. На сегодняшний день их развитие ушло вперёд, однако принципы, заложенные Розенблаттом, являются основополагающими в этой области.

В работе 1963 года \cite{Cepstrum}, которую опубликовали Богерт, Хили и Тьюки, впервые описан кепстр для анализа геологических данных. Позднее этот подход будет использован в анализе речевого сигнала. В десятой главе книги Оппенгейма \cite{CeptrumExplanation} описано его применение в распознавании речи.

В 1967 году впервые применяется спектр для анализа звуковых сигналов \cite{Spectrum}.

В 1971 году появилась одна из первых моделей распознавания большого словаря речевых команд. Она была представлена на конкурсе DARPA и носила название Harpy \cite{Harpy}. Методы, которые были использованы при её реализации актуальны и по сей день.

Для выделения речевых признаков впервые в 1980 году использован алгоритм мел-частотных кепстральных коэффициентов \cite{MFCC}. Он основан на кепстре, который Богерт, Хили и Тьюки использовали для геоанализа.

В 1995 году Ян ЛеКун предложил модель свёрточной нейронной сети для распознавания изображений и речи \cite{CNN}. Все современные архитектуры нейронных сетей в своём большинстве - модификации свёрточной сети.