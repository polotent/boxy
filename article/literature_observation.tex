\specialsection{Обзор литературы}
Первые шаги в распознавании речи были сделаны в 1952 году. Тогда трое исследователей Bell Labs - Стивен Балашек, Рулон Биддалф и Кей Дэвис - представили публике первый в истории аппарат Audrey, способный распознавать человеческую речь \cite{Audrey}. Это была система, позволявшая распознавать только цифры. 

Позднее свои результаты в этой сфере представили компании IBM, AT\&T.


Примерно в 2000 году развитие индустрии приостановилось. К этому моменту устройства были способны распознавать речь с 80\% точностью. Настоящий прорыв произошел, когда компания Google представила свой голосовой поиск. Новшеством было то, что все вычисления были перенесены на мощную серверную часть, а персональные устройства отвечали только за ввод информации. С этого момента все больше компаний стали внедрять распознавание речи в свои продукты.

Среди всех работ в данной задаче стоит отметить:

1) Работа 