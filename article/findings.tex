\specialsection{Выводы}
Видно, что при обучении только на одном дикторе с мужским голосом, точность распознавания на всех остальных дикторах по отдельности низкая. 

При обучении на всех дикторах с мужским голосом, распознавание на дикторе с женским голосом работает лучше, чем при обучении на одном дикторе с мужским голосом, но точность все-равно низкая.

При обучении на всех дикторах, распознавание на каждом даёт приемлемую точность. Однако стоит отметить, что если большая часть дикторов в тренировочной части имеет мужские голоса, то на дикторе с женским голосом распознавание работает хуже, чем на дикторах с мужским голосом.

Можно сделать вывод о том, что свёрточная нейронная сеть немного лучше справляется с поставленной задачей, чем многослойный персептрон. При этом количество весов, влияющее на время распознавания и на объем занимаемой памяти для их хранения, у свёрточной нейронной сети намного меньше из-за особенности её архитектуры.

Предложения по улучшению предложенной модели распознавания речи:
\begin{itemize}[leftmargin=2cm]
	\item увеличить размер датасета для обучения
	\item повысить глубину предложенных архитектур нейронных сетей
	\item увеличить количество коэффициентов MFCC в алгоритме выделения речевых признаков
	\item изменить количество мел-фильтров в алгоритме выделения речевых признаков.
\end{itemize}