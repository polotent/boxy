\specialsection{Введение}
В современном компьютеризированном мире огромное значение имеет взаимодействие человека с компьютером - ввод и вывод информации с устройства в понятной для человека форме. Один из способов внести информацию в компьютер - записать речь через микрофон, после чего можно обрабатывать данные в памяти, которые ее представляют. Обработка может быть совершенно разной, но особенно важно распознавать слова, которые произнес человек и давать им представление в виде текста. То есть давать такое представление речи, как если бы она была не произнесена голосом, а напечатана при помощи клавиатуры. 

Решение такой задачи может быть использовано для различных целей. К примеру, можно переписываться с другим человеком по интернету текстовыми сообщениями, при этом вообще не прикасаясь к клавиатуре, управлять различными компьютерными интерфейсами при помощи голосовых команд и т.д.

Данная проблема была актуальна со времен появления компьютеров и остается таковой и по сей день. Особенно актуальна она стала в последнее время, когда появились качественные микрофоны, возрасла мощность вычислительных устройств, увеличилось количество информации, которой обмениваются люди через интернет.

Изначально, для решения данной задачи применялись такие алгоритмы, как скрытые Марковские модели, методы динамического программирования, методы дискриминантного анализа, основанные на Байесовской дискриминации и другие. Но с появлением нейронных сетей и многочисленных экспериментов с их использованием выяснилось, что задачу распознавания речи можно решать и при помощи нейросетевого подхода. И хоть и сами нейронные сети появились еще в прошлом веке, их популярность возросла только в последнее время в связи с ростом мощности компьютеров.

В данной работе предлагается рассмотреть решение задачи распознавания речи при помощи сверточной нейронной сети. В качестве решения подзадачи выделения характеристик речи предлагается алгоритм мел-частотных кепстральных коэффициентов, разработанный в 70-x годах прошлого века, учитывающий особенности слухового восприятия человеком.

Краткое содержание глав:

В главе 1 рассмотрены теоретические сведения о звуке, речевых признаках и общая схема алгоритма распознавания звука. 
 
В главе 2 рассмотрен алгоритм распознавания речи. Подробно описаны подзадачи предобработки звукового сигнала, выделения речевых признаков и самого блока распознавания алгоритма.

В главе 3 приведены результаты вычислений - результаты обучения и тестирования нейронной сети, а также инструменты, которые были использованы для реализации алгоритма.